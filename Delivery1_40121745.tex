\documentclass{article}
\usepackage[utf8]{inputenc}
\usepackage{array}
\usepackage{wrapfig}
\usepackage{multirow}
\usepackage{graphicx}
\usepackage{tabularx}
\usepackage{geometry}

\makeatletter
%same as \subsubsection but @level 4
\renewcommand\paragraph{\@startsection{paragraph}{4}{\z@}%
{-3.25ex\@plus -lex \@minus -.2ex}%
{1.5ex \@plus .2ex}%
{\normalfont\normalsize\bfseries}}

% number \paragraph
\setcounter{secnumdepth}{4}

\makeatother


\title{\normalsize \texts{SOEN 6481 Summer 2021}\\ [1.0cm]
\large \textbf{\uppercase{Vision Document}}\\
\large \textbf{\uppercase{E-Concordia Drive}}
\normalsize \vspace*{2\baselineskip}\\
\textbf{Disclaimer:}
\textit{"I certify that this submission is the original and meets the Faculty's Expectations of Originality"}
}
%Title and Front page


\author{{Tavtej Singh Lehri}\\
{StudentID - 40121745}}

\begin{document}

\maketitle

\tableofcontents
\clearpage


\section{Introduction}
The purpose of this vision document is to collect and analyze all assorted ideas that have come up to define the system, its requirements with respect to consumers. Also, we shall predict and sort out how we hope this product will be used in order to gain a better understanding of the project, outline concepts that may be developed later, and document ideas that are being considered, but may be discarded as the product develops.\normalsize\vspace*{1\baselineskip}\\
In short, the purpose of this SRS document is to provide a detailed overview of our software product, its parameters and goals. This document describes the project's target audience and its user interface, hardware and software requirements. It defines how our client, team and audience see the product and its functionality. Nonetheless, it helps any designer and developer to assist in software delivery life-cycle (SDLC) processes.\normalsize\vspace*{1\baselineskip}\\
E-Concordia drive is an E-learning platform for students to learn and practice driving lessons to obtain a license. It has got three main users: Trainer, Student and Admin.
Admin edits, comments, approves and publishes  lessons uploaded by a trainer.
Once approved the content is ready to be viewed by students.

\subsection{References}

\end{document}
