\documentclass{article}
\usepackage[utf8]{inputenc}
\usepackage{array}
\usepackage{wrapfig}
\usepackage{multirow}
\usepackage{graphicx}
\usepackage{tabularx}
\usepackage{geometry}


\makeatletter
%same as \subsubsection but @level 4
\renewcommand\paragraph{\@startsection{paragraph}{4}{\z@}%
{-3.25ex\@plus -lex \@minus -.2ex}%
{1.5ex \@plus .2ex}%
{\normalfont\normalsize\bfseries}}

% number \paragraph
\setcounter{secnumdepth}{4}

\makeatother


\title{\normalsize \texts{SOEN 6481 Summer 2021}\\ [1.0cm]
\large \textbf{\uppercase{Vision Document}}\\
\large \textbf{\uppercase{E-Concordia Drive}}
\normalsize \vspace*{2\baselineskip}\\
\textbf{Disclaimer:}
\textit{"I certify that this submission is the original and meets the Faculty's Expectations of Originality"}
}
%Title and Front page


\author{{Tavtej Singh Lehri}\\
{StudentID - 40121745}}

\begin{document}

\maketitle

\tableofcontents
\clearpage


\section{Introduction}
The purpose of this vision document is to collect and analyze all assorted ideas that have come up to define the system, its requirements with respect to consumers. Also, we shall predict and sort out how we hope this product will be used in order to gain a better understanding of the project, outline concepts that may be developed later, and document ideas that are being considered, but may be discarded as the product develops.\normalsize\vspace*{1\baselineskip}\\
In short, the purpose of this SRS document is to provide a detailed overview of our software product, its parameters and goals. This document describes the project's target audience and its user interface, hardware and software requirements. It defines how our client, team and audience see the product and its functionality. Nonetheless, it helps any designer and developer to assist in software delivery life-cycle (SDLC) processes.\normalsize\vspace*{1\baselineskip}\\
E-Concordia drive is an E-learning platform for students to learn and practice driving lessons to obtain a license. It has got three main users: Trainer, Student and Admin.
Admin edits, comments, approves and publishes  lessons uploaded by a trainer.
Once approved the content is ready to be viewed by students.

\subsection{References}


\section{Positions}

\subsection{Problem Statement}

\begin{table}[h!]
\begin{tabular}{|p{4.5cm}|p{11.5cm}|}
\hline
\textbf{The Problem of} & Not being able to hold in-person learning and practice for obtaining driver's license due to COVID-19 \\ \hline
\textbf{Affects} & All people who are new to driving and existing course students. \\ \hline
\textbf{The impact of which is} & That due to the pandemic the driving school had to close the in-person classes. \\ \hline
\textbf{The Solution to which is} & Creating an E-Learning website using which, the students can watch training videos prepared by instructors. Also based on the training module students can solve quizzes, which will help them practice for the main exam.\\ \hline
\end{tabular}
\caption{Problem Statement}
\label{table:1}
\end{table}

\subsection{Product Position Statement}

\begin{table}[h!]
\begin{tabular}{|p{4.5cm}|p{11.5cm}|}
\hline
\textbf{For} & All people who are new to driving and existing course students\\ \hline
\textbf{Who}& Want to practice for the final driving exam.\\ \hline
\textbf{The E-Concordia Drive} & is an e-learning web product\\ \hline
\textbf{That} & helps students prepare for their final writing test, while sitting at their home.\\ \hline
\textbf{Unlike} & attending school amidst of the pandemic \\ \hline
\textbf{Our Product} & Helps the student learn from the vicinity of their homes. Using this platform they have access to videos 24/7, which helps revisit the videos them with if they did not understand something at first. Apart from this the platform also helps with practice quizzes.\\ \hline
\end{tabular}
\caption{Product Positioning Statement}
\label{table:2}
\end{table}

\section{Stakeholder Description}

\subsection{Stakeholder Summary}

\begin{table}[h!]
\begin{tabular}{|p{4.5cm}|p{4.5cm}|p{6.5cm}|}
\hline
\textbf{For} & \textbf{Description} & \textbf{Responsibilities} \\ \hline
Who & What & \\ \hline
The E-Concordia Drive & Students &\\ \hline
That & This &\\ \hline
Unlike & Dislike &\\ \hline
Our Product & Their Product &\\ \hline
\end{tabular}
\caption{Stakeholders Summary}
\label{table:3}
\end{table}


\subsection{User Summary}

\begin{table}[h!]
\begin{tabular}{|p{3.5cm}|p{3.5cm}|p{4.5cm}|p{4cm}|}
\hline
\textbf{Name} & \textbf{Description} & \textbf{Responsibilities} & \textbf{Stakeholder}\\ \hline
Who & What & & \\ \hline

\end{tabular}
\caption{User Summary - Students}
\label{table:4}
\end{table}

\subsection{User Environment}

\subsection{Key Stakeholder or Users Needs}

\begin{table}[h!]
\begin{tabular}{|p{2.5cm}|p{2.5cm}|p{3.5cm}|p{3.5cm}|p{3.5cm}|}
\hline
\textbf{Need} & \textbf{Priority} & \textbf{Concerns} & \textbf{Current Solution} & \textbf{Proposed Solution}\\ \hline
Who & What & & &\\ \hline

\end{tabular}
\caption{User Summary - Students}
\label{table:5}
\end{table}

\section{Product Overview}

\subsection{Product Perspectives}

\subsection{Assumptions and Dependencies}

\begin{table}[h!]
\begin{center}
\begin{tabular}{|p{5.5cm}|p{6.5cm}|}
\hline
\textbf{Assumptions} & \textbf{Dependencies} \\ \hline
Who & What\\ \hline

\end{tabular}
\caption{User Summary - Students}
\label{table:6}
\end{center}
\end{table}


\section{Product Features}

\section{Other Product Requirements}

\end{document}
