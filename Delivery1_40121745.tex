\documentclass{article}
\usepackage[utf8]{inputenc}
\usepackage{array}
\usepackage{wrapfig}
\usepackage{multirow}
\usepackage{graphicx}
\usepackage{tabularx}
\usepackage{geometry}
\usepackage{changepage}


\makeatletter
%same as \subsubsection but @level 4
\renewcommand\paragraph{\@startsection{paragraph}{4}{\z@}%
{-3.25ex\@plus -lex \@minus -.2ex}%
{1.5ex \@plus .2ex}%
{\normalfont\normalsize\bfseries}}

% number \paragraph
\setcounter{secnumdepth}{4}

\makeatother


\title{\normalsize \texts{SOEN 6481 Summer 2021}\\ [1.0cm]
\large \textbf{\uppercase{Vision Document}}\\
\large \textbf{\uppercase{E-Concordia Drive}}
\normalsize \vspace*{2\baselineskip}\\
\textbf{Disclaimer:}
\textit{"I certify that this submission is the original and meets the Faculty's Expectations of Originality"}
}
%Title and Front page


\author{{Tavtej Singh Lehri}\\
{StudentID - 40121745}}

\begin{document}

\maketitle

\tableofcontents
\clearpage


\section{Introduction}
The purpose of this vision document is to collect and analyze all assorted ideas that have come up to define the system, its requirements with respect to consumers. Also, we shall predict and sort out how we hope this product will be used in order to gain a better understanding of the project, outline concepts that may be developed later, and document ideas that are being considered, but may be discarded as the product develops.\normalsize\vspace*{1\baselineskip}\\
In short, the purpose of this SRS document is to provide a detailed overview of our software product, its parameters and goals. This document describes the project's target audience and its user interface, hardware and software requirements. It defines how our client, team and audience see the product and its functionality. Nonetheless, it helps any designer and developer to assist in software delivery life-cycle (SDLC) processes.\normalsize\vspace*{1\baselineskip}\\
E-Concordia drive is an E-learning platform for students to learn and practice driving lessons to obtain a license. It has got three main users: Trainer, Student and Admin.
Admin edits, comments, approves and publishes  lessons uploaded by a trainer.
Once approved the content is ready to be viewed by students.

\subsection{References}


\section{Positions}

\subsection{Problem Statement}

\begin{table}[h!]
\begin{tabular}{|p{4.5cm}|p{11.5cm}|}
\hline
\textbf{The Problem of} & Not being able to hold in-person learning and practice for obtaining driver's license due to COVID-19 \\ \hline
\textbf{Affects} & All people who are new to driving and existing course students. \\ \hline
\textbf{The impact of which is} & That due to the pandemic the driving school had to close the in-person classes. \\ \hline
\textbf{The Solution to which is} & Creating an E-Learning website using which, the students can watch training videos prepared by instructors. Also based on the training module students can solve quizzes, which will help them practice for the main exam.\\ \hline
\end{tabular}
\caption{Problem Statement}
\label{table:1}
\end{table}

\subsection{Product Position Statement}


\begin{tabular}{|p{4.5cm}|p{11.5cm}|}
\hline
\textbf{For} & All people who are new to driving and existing course students\\ \hline
\textbf{Who}& Want to practice for the final driving exam.\\ \hline
\textbf{The E-Concordia Drive} & is an e-learning web product\\ \hline
\textbf{That} & helps students prepare for their final writing test, while sitting at their home.\\ \hline
\textbf{Unlike} & attending school amidst of the pandemic \\ \hline
\textbf{Our Product} & Helps the student learn from the vicinity of their homes. Using this platform they have access to videos 24/7, which helps revisit the videos them with if they did not understand something at first. Apart from this the platform also helps with practice quizzes.\\ \hline
\end{tabular}

\section{Stakeholder Description}

\subsection{Stakeholder Summary}


\begin{tabular}{|p{4.5cm}|p{4.5cm}|p{6.5cm}|}
\hline
\textbf{Name} & \textbf{Description} & \textbf{Responsibilities} \\ \hline
Business Analyst & Is a person responsible for the SRS document and understanding the business needs & Should prepare SRS document and any changes to it, understand the business requirements, understand the customer requirements, layout clear and precise requirements for the developers and testers.\\ \hline
Architects & Acts as development lead and is a guy who has knowledge of MVC architecture  & Create the MVC style architecture for the developers, communicate with business and clients to design and execute solutions. Make executive software design decisions\\ \hline
Developers & Coders of the website & They are responsible to write a stable, understandable and easily executable PHP language based website. Should be familiar with MVC architecture. Should be able to read and understand requirements from the requirement document\\ \hline
Database Admins & Is a person who maintains and develops database for the application. & Should be familiar with MySQL database. Responsible for creating and maintaining the database. Should have in-depth knowledge of SQL.\\ \hline
QA/Test Analyst & Is a person who can find bugs in the application and help achieve stable application & Should be familiar with JIRA, can create test suite, test plan, test cases.\\ \hline
Marketing \& Advertising & A firm responsible to bring the application to end user. & Select the target users, advertise to the users, maintain social media advertising and help in SEO. \\ \hline
\end{tabular}


\subsection{User Summary}

\begin{tabular}{|p{3.5cm}|p{3.5cm}|p{4.5cm}|p{4cm}|}
\hline
\textbf{Name} & \textbf{Description} & \textbf{Responsibilities} & \textbf{Stakeholder}\\ \hline
Students &  & & \\ \hline
Teachers &  & & \\ \hline
Admin &  & & \\ \hline
Students &  & & \\ \hline
Students &  & & \\ \hline

\end{tabular}

\subsection{User Environment}
\begin{itemize}
    \item\textbf{Tools used for the project:}
    \begin{itemize}
        \item \textbf{PhpStorm:} IDE for software development
        \item \textbf{JIRA:} To improve the management of new development and defects.
        \item \textbf{GitHub:} for configuration management of the source code, test code and data.
        \item \textbf{MySQL:} for creating and maintaining the data.
        \item \textbf{Overleaf:} for creating all project related documentations
    \end{itemize}
    \item\textbf{Team Structure:}
    \begin{itemize}
        \item There will be 1 software architect who will also act as the development manager. The development team will consist of 4-5 developers and will be lead by the architect. Number of developers will go down to 2 once the project is live.
        \item The testing team will have 4 testers. The 4 people includes a test lead and 3 test analysts. Number of testers will reduce to 1-2 once the project goes live.
        \item There will be 1 scrum master and 1 business analyst.
        \item 1 developer and 1 tester will be moved to maintenance team once the project is live and deal wil production issues. 
    \end{itemize}
    \item Development of the product will be done using agile methodologies. It will be divided in 3 iterations. Iteration 1 will be 8 weeks long, Iterations 2 and 3 each will be 4 weeks long.
    \item For this release the project will work on Google Chrome and Safari. Expansion of this project will see working on other web browsers available in the market. Also there is a plan to create an application after analyzing the response for the website.
    \item Users should have a browser enabled device for using the website. Also the devices they use should be capable of viewing video and audio clips.
    \item Marketers and Advertisers should be familiar with the SEO and SCO tools. They should track all the social media activities via twitter, facebook and instagram.
    \item Student users shall have access to all the videos and quizzes once they enroll for the course.
    \item Trainer users should be able to record, manage and publish the videos for the approval of admin. Also they should be able to create quizzes for students to practice.
    \item Admin shall have the access to approve/reject the published stuff by the trainers.
\end{itemize}

\subsection{Key Stakeholder or Users Needs}


\begin{tabular}{|p{2.5cm}|p{2.5cm}|p{3.5cm}|p{3.5cm}|p{3.5cm}|}
\hline
\textbf{Need} & \textbf{Priority} & \textbf{Concerns} & \textbf{Current Solution} & \textbf{Proposed Solution}\\ \hline
Who & What & & &\\ \hline

\end{tabular}


\section{Product Overview}

\subsection{Product Perspectives}

\subsection{Assumptions and Dependencies}

\begin{tabular}{|p{5.5cm}|p{6.5cm}|}
\hline
\textbf{Assumptions} & \textbf{Dependencies} \\ \hline
Who & What\\ \hline

\end{tabular}



\section{Product Features}

\section{Other Product Requirements}

\end{document}
